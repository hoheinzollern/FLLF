\documentclass[a4paper,10pt]{article}

\usepackage{ucs}
\usepackage[utf8x]{inputenc}
\usepackage{amsmath}
\usepackage{amsfonts}
\usepackage{amssymb}
\usepackage{amsthm}
\usepackage[english]{babel}
\usepackage{fontenc}
\usepackage{graphicx}

\usepackage[dvips]{hyperref}

\author{Silvio Valentini\\
Dipartimento di Matematica Pura ed Applicata\\
Università di Padova\\
via G. Belzoni n.7, I-35131 Padova, Italy\\
silvio@math.unipd.it}
\title{Pure Lambda Calculus Models in Applicative Structures}
\date{April 27, 2004}

\newtheorem{definition}{Definition}
\newtheorem{theorem}{Theorem}
\newtheorem{lemma}{Lemma}

\begin{document}

\maketitle

\begin{abstract}
 Just an introduction for students to the models of pure lambda calculus in
applicative structures.
\end{abstract}

\section{Combinatoric completeness}
The basic definition just gives a name to one of the most general situation in
mathematics.

\begin{definition} Let $A$ be a set, $\cdot$ be a binary operation over $A$,
i.e., for any $a,b \in A,\, a \cdot b \in A$, and $=_A$ an equivalence relation
between elements of $A$ which respects $\cdot$. Then the structure $(A, \cdot,
=_A)$ is called an \emph{applicative structure} over $A$ and the operation
$\cdot$ an \emph{application}.
\end{definition}

Note that no property is required on the operation $\cdot$; in the following,
in order to make the notation simpler we will write $a_1 \cdot a_2 \cdot \dots
\cdot a_n$ to mean $((a_1 \cdot a_2) \cdot \dots \cdot a_n)$ and we will add
all the needed parenthesis to avoid any ambiguity.

It is clear that almost every mathematical structure can be thought of like an
applicative structure. This is the reason why applicative structure are not so
interesting for us in the most general case; but a sub-class of them can indeed
be interesting. First, let us define the notion of terms over $A$.

\begin{definition}
The set $T(A)$ of terms over $A$ is inductively defined by the following rules
$$T(A) \equiv Var(T) | A | T(A) \cdot T(A)$$
\end{definition}

Thus terms over $A$ are nothing else than monomial over $A$. We will use the
notation $t[x_1 \dots x_n]$ to mean a term over $A$ whose variable are among
$x_1 \dots x_n$ and $t[a_1 \dots a_n]$ to mean the substitution of the
variables appearing in $t$ by elements of $A$, i.e. $t[a_1 \dots a_n \equiv
t[x_1 \dots x_n][x_1:=a_1 \dots x_n:=a_n]$.

We can easily extend the equivalence relation $=_A$ to an equivalence relation
between terms over $A$ by putting
$$t_1[x_1 \dots x_n] =_{T(A)} t_2[x_1 \dots x_n] \, \text{iff} \, \forall a_1
\dots a_n \in A \,.\, t_1[a_1 \dots a_n] =_A t_2[a_1 \dots a_n]$$

It is immediate to verify that $=_{T(A)}$ is an equivalence relation between
terms over $A$.

Finally we give the following definition.

\begin{definition}
 An applicative structure $(A, \cdot, =_A)$ is \emph{functionally complete} if
and only if for any term $t[x_1 \dots x_n]$ over $A$ there exists an element
$f \in A$ such that
$$f \cdot x_1 \cdot \dots \cdot x_n =_{T(A)} t[x_1 \dots x_n]$$
\end{definition}

The property of functional completenes is so much interesting that the
applicative structures which enjoies it deserve a special name and are called
\emph{combinatory algebras}. Now we can state the main result in this section
which caracterize combinatory algebras among applicative structures.

\begin{theorem}
 An applicative structure $(A, \cdot, =_A)$ is functionally complete if and
only if there exist two elements $K$ and $S$ in $A$ such that
$$K \cdot x \cdot y =_{T(A)} x \hspace{2em} S \cdot x \cdot y \cdot z =_{T(A)}x
\cdot z \cdot (y \cdot z)$$
\end{theorem}

One implication is immediate. In fact, suppose the applicative structure is
functionally complete and consider the terms $t_K[x,y] \equiv x$ and
$t_S[x,y,z] \equiv x \cdot z \cdot (y \cdot z)$. Then the elements $K$ and $S$
respectively associated to the terms $t_K$ and $t_S$ obviously satisfy the
required conditions. The other implication is not immediate but we can simplify
its proof if we show first the following lemma.

\begin{lemma}\label{lemma1}
Let $(A, \cdot, =_A)$ be an applicative structure which contains two elements
$K$ and $S$ such that
$$K \cdot x \cdot y =_{T(A)} x \hspace{2em} S \cdot x \cdot y \cdot z =_{T(A)}x
\cdot z \cdot (y \cdot z)$$

Then, for any term $t[x_1 \dots x_n]$ there exists a term $f[x_1 \dots x_n-1]$
such that
$$f[x_1 \dots x_n-1] \cdot x_n =_{T(A)} t[x_1 \dots x_n]$$
\end{lemma}

In fact, supposing that the previous lemma holds, the main theorem can be
proved by a simple induction on the number of variables $x_1, \dots ,x_n$ in
$t$ as follows:

\begin{align*}
t[x_1 \dots x_n] =_{T(A)}& f_1[x_1 \dots x_{n-1}] \cdot x_n\\
=_{T(A)}& f_2[x_1 \dots x_{n-2}] \cdot x_{n-1} \cdot x_n\\
& \dots\\
=_{T(A)}& f_n \cdot x_1 \cdot \dots \cdot x_{n-1} \cdot x_n
\end{align*}

In order to prove Lemma~\ref{lemma1} we will make an analysis by cases on the
possible shape of the term $t$:

\begin{itemize}
 \item $t[x_1 \dots x_n] \equiv x_n$.

Then put $f[x_1 \dots x_{n-1}] \equiv I \equiv S \cdot K \cdot K$. In fact,

\begin{align*}
 f[x_1 \dots x_{n-1}] \cdot x_n \equiv & I \cdot x_n\\
\equiv & S \cdot K \cdot K \cdot x_n\\
= & K \cdot x_n \cdot (K \cdot x_n)\\
= & x_n
\end{align*}

\item $t[x_1 \dots x_n] \equiv x_i$ for some $i < n$.

Then put $f[x_1 \dots x_{n-1}] \equiv K \cdot x_i$. In fact,

\begin{align*}
f[x_1 \dots x_{n-1}] \cdot x_n \equiv & K \cdot x_i \cdot x_n\\
= & x_i
\end{align*}

\item $t[x_1 \dots x_{n-1}] \equiv a$ for some $a \in A$.

Then put $f[x_1 \dots x_{n-1}] \equiv K \cdot a$. In fact,

\begin{align*}
f[x_1 \dots x_{n-1}]\cdot x_n \equiv & K \cdot a \cdot x_n\\
= & a
\end{align*}

\item $t[x_1 \dots x_n] \equiv t_1 [x_1 \dots x_n] \cdot t_2 [x_1 \dots x_n]$.

Then put $$ f[x_1 \dots x_{n-1}] \equiv S \cdot f_1[x_1 \cdots x_{n-1}] \cdot
f_2[x_1 \cdots x_{n-1}]$$ where $f_1[x_1 \cdots x_{n-1}]$ is the term
associated with the term $t_1[x_1 \cdots x_{n}]$ and $f_2[x_1 \cdots x_{n-1}]$
is the term associated with the term $t_2[x_1 \cdots x_{n}]$. In fact,

\begin{align*}
f[x_1 \dots x_{n-1}] \equiv & S \cdot f_1[x_1 \cdots x_{n-1}] \cdot f_2[x_1
\cdots x_{n-1}] \cdot x_n\\
= & f_1[x_1 \cdots x_{n-1}] \cdot x_n \cdot (f_2[x_1 \cdots x_{n-1}] \cdot
x_n)\\
= & t_1[x_1 \cdots x_{n}] \cdot t_2[x_1 \cdots x_{n}]
\end{align*}

\end{itemize}

\section{Interpretation of pure lambda-calculus}

We will show now how to interpret the lambda terms in the set $\Lambda$
inductively defined by the following rules:
$$\Lambda \equiv Var_\Lambda | \Lambda(\Lambda) | \lambda x . \Lambda$$
Into elements of any combinatory algebra $(A, \cdot, K, S, =_A)$.

First, note that the combinatory algebra is functionally complete by the
previous Theorem. Thus, for any term $t[x_1 \dots x_n, x]$ there is a term
$f[x_1 \dots x_n]$ such that $f[x_1 \dots x_n] \cdot x = t[x_1 \dots x_n, x]$.
We will indicate the term $f[x_1 \dots x_n]$ by $\Lambda x . t$.

Then we want to define an interpretation of the elements of $\Lambda$ into
elements of $A$. To this aim let us suppose to have a map $\rho : Var_\Lambda
\rightarrow A$ and let us extend it to a new map $\rho^* : T(A) \rightarrow A$,
where $T(A)$ is the set of terms over $A$ built by using the variables in
$Var_\Lambda$, by putting

\begin{align}
\rho^*(x) \equiv & \rho(x)\\
\rho^*(a) \equiv & a\\
\rho^*(t \cdot u) \equiv & \rho^*(t) \cdot \rho^*(u)
\end{align}

Then, we can go on as follows. Let $[-]:\Lambda \rightarrow T(A)$ be the
following map between lambda terms and terms in $T(A)$.

\begin{align}
[x] \equiv & x \\
[M(N)] \equiv & [M] \cdot [N] \\
[\lambda x \cdot M] \equiv & \Lambda x \cdot [M]
\end{align}

Now we can obtain the wished interpretation by composing $\rho^*$ and $[-]$ as
follows:
$$[M]_\rho \equiv \rho^*([M])$$
and we obtain a map from lambda terms into elements of $A$.

\end{document}
